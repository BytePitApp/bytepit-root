\chapter{Zaključak i budući rad}
		
		Razumijevajući postavljene zahtjeve ovog projekta, 
		pristupili smo zadatku s dubokim poštovanjem prema predmetu i tradicionalnim 
		metodama, što je uključivalo stvaranje opsežne dokumentacije 
		i brojnih UML dijagrama. Međutim, kroz ovaj proces, naš tim je stekao dublje razumijevanje o ograničenjima i nedostacima ovakvog pristupa, posebno kada se uspoređuje s agilnijim i fleksibilnijim metodama koje su danas prisutne u industriji softverskog inženjerstva.
		
		Naša iskustva s ovim projektom jasno su istaknula zašto je u posljednjih deset godina industrija napustila praksu pisanja opsežne dokumentacije. Iako smo pažljivo i precizno kreirali UML dijagrame i detaljnu dokumentaciju, često smo se suočavali s teškoćama koje su proizašle iz njihove inherentne rigidnosti. Ova vrsta dokumentacije ograničava mogućnost brze prilagodbe promjenama, što je od ključne važnosti u brzom i dinamičnom tehnološkom okruženju.
		
		UML dijagrami, iako korisni za vizualizaciju i planiranje, mogu biti (i bili su) prekomjerni i često ne odražavaju stvarnu kompleksnost ili fluidnost razvojnog procesa. Naša iskustva su pokazala da ovakva vrsta dokumentacije može biti kontraproduktivna, jer zahtijeva značajne količine vremena i resursa koji bi se mogli bolje iskoristiti u stvarnom razvoju proizvoda.
		
		U projektu koji traje 3 mjeseca, vrijeme utrošeno na crtanje UML dijagrama je vrijeme koje kasnije nedostaje da se projekt u potpunosti izvrši. Takav pristup se donekle još može opravdati u projektima koji traju godinama i koje osmišljaju najbolji softverski arhitekti s godinama iskustva. Međutim, u našem slučaju, gdje je cilj bio stvoriti funkcionalan proizvod u kratkom roku, ovakav pristup je bio kontraproduktivan i ograničavajući. Gotovo niti jedan aspekt koji je planiran UML dijagramima na kraju nije napravljen na taj način, već je u procesu razvoja došlo do prilagodbi, a zastarjeli dijagrami su se nanovo morali korigirati.

		S druge strane, Agile pristup (ili njegove manje dosljedne adaptacije), koji smo kritički promatrali kroz prizmu našeg projekta, nudi alternativu koja se fokusira na adaptabilnost, iterativni razvoj i stalnu suradnju. Agile priznaje da se pravi uspjeh u razvoju softvera ne postiže kroz statične planove i dokumente, već kroz neprekidno prilagođavanje, učenje i suradnju. Agile metodologija potiče timove na brzu reakciju na promjene i na fokusiranje na stvaranje proizvoda koji odgovara trenutnim potrebama tržišta i korisnika.
		
		Kroz naš projekt, usmjerili smo fokus prema stvaranju zapravo korisnih elemenata, koji su u suštini temelj modernog softverskog razvoja. Ovaj pristup je uključivao razvoj i implementaciju Continuous Integration/Continuous Deployment (CI/CD) pipelineova, korištenje alata kao što su Jira i Confluence za upravljanje projektima i dokumentacijom, te usvajanje automatizacije u različitim aspektima našeg rada.

		CI/CD pipelineovi su bili ključni za našu sposobnost brzog i efikasnog razvoja softvera. Kroz automatizaciju procesa builda, testiranja i deploymenta, uspjeli smo znatno ubrzati razvojni ciklus i povećati kvalitetu našeg koda. Ovo je omogućilo timu da se brzo prilagodi promjenama, smanjujući vrijeme potrebno za implementaciju novih funkcionalnosti i popravak bugova.

		Korištenje Jire i Confluencea je dodatno pojačalo našu sposobnost efikasnog upravljanja projektom. Jira nam je omogućila da učinkovito pratimo napredak i upravljamo zadacima, dok je Confluence poslužio kao centralno mjesto za pohranu i organizaciju naše projektne dokumentacije. Ovaj integrirani pristup upravljanju projektima i dokumentacijom omogućio nam je da ostanemo organizirani, transparentni i usmjereni na ciljeve projekta.

		Ovi elementi projekta, iako manje vidljivi u tradicionalnoj dokumentaciji, bili su vitalni za naš uspjeh i efikasnost. Dok smo se pridržavali zahtjeva za opsežnom dokumentacijom i UML dijagramima, naša stvarna vrijednost ležala je u primjeni ovih naprednih alata i tehnika. Naš rad demonstrira kako, unatoč poštivanju tradicionalnih zahtjeva, možemo uspješno integrirati i koristiti moderne alate i prakse koje su sada standard u industriji.

		U retrospektivi, naše iskustvo s ovim projektom je bilo dragocjeno učenje o važnosti balansa između planiranja i fleksibilnosti. Dok smo uspješno ispunili zahtjeve za detaljnom dokumentacijom i UML dijagramima, također smo stekli uvid u njihova ograničenja i razloge zašto suvremena industrija softverskog inženjerstva preferira agilnije pristupe. Kritički osvrt na ovaj projekt ne umanjuje vrijednost rada koji smo obavili, već osvjetljava put ka efikasnijim i prilagodljivijim metodama koje su danas relevantnije u dinamičkom tehnološkom svijetu.
		
		S obzirom da ne postoji predviđena sekcija u dokumentaciji za to, ovdje ćemo navesti neke statistike vezane uz projekt:
		\begin{packed_item}
			\item 177 pull requestova
			\item 175 Jira ticketa
			\item 200+ CI/CD minuta
		\end{packed_item}
		\eject 