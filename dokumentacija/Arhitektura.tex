\chapter{Arhitektura i dizajn sustava}
	\textit{Arhitektura cijelog sustava može se podijeliti na četiri ključna dijela:}
		\begin{itemize}
    		\item Web preglednik
    		\item Web aplikacija
    		\item Baza podataka
    		\item Code runner
		\end{itemize}

	\textit{\underline{Web preglednik} omogućuje interakciju između korisnika i aplikacije. Svaki se korisnikov zahtjev
	 događa na web pregledniku i prosljeđuje aplikaciji na obradu. Osim slanja zahtjeva, korisniku se omogućuje
	  bolji pregled aplikacije pomoću apstraktnog koda kojeg web preglednik pretvara u lako shvatljive strukture.}\\

	\textit{\underline{Aplikacija} se sastoji od dva glavna dijela, frontenda i backenda. Tehnologije koje se koriste na backendu
	 uključuju programski jezik \textbf{Python} i \textbf{fastApi} web framework koji omogućuje stvaranje aplikacije temeljene na \textbf{REST}
	 arhitekturi i rutera na kojem su definirane rute za slanje zahtjeva i komunikaciju korisnika i aplikacije.
	 Frontend aplikacije izrađen je korištenjem JavaScript librarya React. On omogućuje korisniku jednostavan pregled
	 aplikacije i jednostavno korištenje aplikacije, odnosno slanje zahtjeva.}\\

	\textit{\underline{Kontejneri} se koriste unutar aplikacije kako bi poboljšali korisničko iskustvo, optimizirali performanse sustava
	te osigurali skalabilnost za povećani broj korisnika. Glavni se dio aplikacije s ruterima i logikom za komunikaciju s 
	bazom nalazi u vlastitom kontejneru kao i glavna baza te evaluator.}\\

	\textit{Za \underline{bazu podataka} se koristi \textbf{SQL} specifično \textbf{PostgreSQL}. Glavna baza se nalazi u vlastitom kontejneru kojemu backend
	aplikacije pristupa kako bi spremio informacije u nju. Na backendu se koristi \textbf{psycopg} PostgreSQL adapter koji omogućuje
	komunikaciju između baze i backenda.}\\
	
	\textit{\underline{Evaluator} komunicira s vanjskim \underline{Code runnerom} za kompajliranje kodova koje korisnik šalje
	na backend. Ovaj ključni dio sustava odgovoran je za izvršavanje kôda unutar sigurnog okruženja, čime se osigurava
	izolacija korisničkih skripti od ostatka aplikacije. Kroz ovu komunikaciju, Evaluator prima kôd od korisnika,
	proslijeđuje ga Code runneru na izvršavanje te zatim interpretira rezultate kako bi ih pravilno integrirao
	natrag u aplikaciju.}
	
	\newpage
	
		

		

				
		\section{Baza podataka}
			
			\textbf{\textit{dio 1. revizije}}\\
			
		\textit{Potrebno je opisati koju vrstu i implementaciju baze podataka ste odabrali, glavne komponente od kojih se sastoji i slično.}
		
			\subsection{Opis tablica}
			

				\textit{Svaku tablicu je potrebno opisati po zadanom predlošku. Lijevo se nalazi točno ime varijable u bazi podataka, u sredini se nalazi tip podataka, a desno se nalazi opis varijable. Svjetlozelenom bojom označite primarni ključ. Svjetlo plavom označite strani ključ}
				
				
				\begin{longtblr}[
					label=none,
					entry=none
					]{
						width = \textwidth,
						colspec={|X[6,l]|X[6, l]|X[20, l]|}, 
						rowhead = 1,
					} %definicija širine tablice, širine stupaca, poravnanje i broja redaka naslova tablice
					\hline \SetCell[c=3]{c}{\textbf{korisnik - ime tablice}}	 \\ \hline[3pt]
					\SetCell{LightGreen}IDKorisnik & INT	&  	Lorem ipsum dolor sit amet, consectetur adipiscing elit, sed do eiusmod  	\\ \hline
					korisnickoIme	& VARCHAR &   	\\ \hline 
					email & VARCHAR &   \\ \hline 
					ime & VARCHAR	&  		\\ \hline 
					\SetCell{LightBlue} primjer	& VARCHAR &   	\\ \hline 
				\end{longtblr}
				
				
			
			\subsection{Dijagram baze podataka}
				\textit{ U ovom potpoglavlju potrebno je umetnuti dijagram baze podataka. Primarni i strani ključevi moraju biti označeni, a tablice povezane. Bazu podataka je potrebno normalizirati. Podsjetite se kolegija "Baze podataka".}
			
			\eject
			
			
		\section{Dijagram razreda}
		
			\textit{Potrebno je priložiti dijagram razreda s pripadajućim opisom. Zbog preglednosti je moguće dijagram razlomiti na više njih, ali moraju biti grupirani prema sličnim razinama apstrakcije i srodnim funkcionalnostima.}\\
			
			\textbf{\textit{dio 1. revizije}}\\
			
			\textit{Prilikom prve predaje projekta, potrebno je priložiti potpuno razrađen dijagram razreda vezan uz \textbf{generičku funkcionalnost} sustava. Ostale funkcionalnosti trebaju biti idejno razrađene u dijagramu sa sljedećim komponentama: nazivi razreda, nazivi metoda i vrste pristupa metodama (npr. javni, zaštićeni), nazivi atributa razreda, veze i odnosi između razreda.}\\
			
			\textbf{\textit{dio 2. revizije}}\\			
			
			\textit{Prilikom druge predaje projekta dijagram razreda i opisi moraju odgovarati stvarnom stanju implementacije}
			
			
			
			\eject
		
		\section{Dijagram stanja}
			
			
			\textbf{\textit{dio 2. revizije}}\\
			
			\textit{Potrebno je priložiti dijagram stanja i opisati ga. Dovoljan je jedan dijagram stanja koji prikazuje \textbf{značajan dio funkcionalnosti} sustava. Na primjer, stanja korisničkog sučelja i tijek korištenja neke ključne funkcionalnosti jesu značajan dio sustava, a registracija i prijava nisu. }
			
			
			\eject 
		
		\section{Dijagram aktivnosti}
			
			\textbf{\textit{dio 2. revizije}}\\
			
			 \textit{Potrebno je priložiti dijagram aktivnosti s pripadajućim opisom. Dijagram aktivnosti treba prikazivati značajan dio sustava.}
			
			\eject
		\section{Dijagram komponenti}
		
			\textbf{\textit{dio 2. revizije}}\\
		
			 \textit{Potrebno je priložiti dijagram komponenti s pripadajućim opisom. Dijagram komponenti treba prikazivati strukturu cijele aplikacije.}