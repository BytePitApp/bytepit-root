\chapter{Dnevnik promjena dokumentacije}
		
		\textbf{\textit{Kontinuirano osvježavanje}}\\
				
		
		\begin{longtblr}[
				label=none
			]{
				width = \textwidth, 
				colspec={|X[2]|X[13]|X[3]|X[3]|}, 
				rowhead = 1
			}
			\hline
			\textbf{Rev.}	& \textbf{Opis promjene/dodatka} & \textbf{Autori} & \textbf{Datum}\\[3pt] \hline
			0.1 & Napravljen predložak.	& Marko Bolt & 04.11.2023. 		\\[3pt] \hline 
			0.2 & Napravljen opis projekta & Marko Bolt & 05.11.2023.\\[3pt] \hline	
			0.3 & Napravljeni funkcionalni zahtjevi & Jakov Vinožganić Teo Radolović & 09.11.2023.\\[3pt] \hline	
			0.4 & Napravljeni obrasci uporabe & Filip Bernt & 10.11.2023.\\[3pt] \hline
			0.5 & Napravljeni sekvencijski dijagrami & Filip Bernt Fran Sipić & 10.11.2023.\\[3pt] \hline	
			0.6 & Napravljeni ostali zahtjevi & Jure Franjković & 12.11.2023.\\[3pt] \hline	
			0.7 & Napravljena arhitektura i dizajn sustava & Jure Franjković & 13.11.2023.\\[3pt] \hline	
			0.8 & Napravljen opis baze podataka & Filip Bernt Fran Sipić & 14.11.2023.\\[3pt] \hline	
			0.9 & Napravljen dijagram razreda & Jure Franjković & 15.11.2023.\\[3pt] \hline	
			1.0 & Napravljena finalna inačica & Filip Bernt & 17.11.2023.\\[3pt] \hline	
		\end{longtblr}
	
	
		\textit{Moraju postojati glavne revizije dokumenata 1.0 i 2.0 na kraju prvog i drugog ciklusa. Između tih revizija mogu postojati manje revizije već prema tome kako se dokument bude nadopunjavao. Očekuje se da nakon svake značajnije promjene (dodatka, izmjene, uklanjanja dijelova teksta i popratnih grafičkih sadržaja) dokumenta se to zabilježi kao revizija. Npr., revizije unutar prvog ciklusa će imati oznake 0.1, 0.2, …, 0.9, 0.10, 0.11.. sve do konačne revizije prvog ciklusa 1.0. U drugom ciklusu se nastavlja s revizijama 1.1, 1.2, itd.}